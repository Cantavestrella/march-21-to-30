
\documentclass[12pt]{article}

% --------------------------------------------------------

\input{../preamble}

\renewcommand{\espaiAbansEtiquetaPoema}{\vspace{0ex}}
\renewcommand{\espaiAbansSeccio}{\vspace{0ex}}

% --------------------------------------------------------

\begin{document}


\begin{estrofa}

\espaiAbansEtiquetaPoema

\poema{(d 6) [21]}\\

\begin{rubrica}

\pagina{[12r]} DE AMOR CANTO. VI.

\end{rubrica}

\end{estrofa}


\begin{estrofa}

\numero{1} \textsc{Ta}nt en Amor / ma pens'ha consentit

\numero{2} q\textit{ue} s\'{e}ns aquell / en als no puch ente\textit{n}dre

\numero{3} ami quem plau / q\textit{ue }d'als no pusch aprendre

\numero{4} tot altre fer / mi entre'n gran despit.

\numero{5} Los grans thesors / ne tot l'honor del mon

\numero{6} nom plau hauer / ab menys de ser amat

\numero{7} car sens a\c{c}o / nom do ben ahuirat

\numero{8} \c{c}o que no es car / tot mon desig con fon.

\end{estrofa}



\begin{estrofa}

\numero{9} Pobre so donchs / molt mes que Iob no fo\textit{n}

\numero{10} puis es dit rich / \c{c}ell qui no ha desig:

\numero{11} en passions / yom trob dins en lo mig

\numero{12} si desijar / ab desesper l'hom fon.

\numero{13} Iames Amor / fon axi auinent

\numero{14} en be mostrar /sa famosa virtut!

\numero{15} com al muntar / me: y ferme d'abatut,

\numero{16} ben ahuirat / sus tots complidament.

\end{estrofa}



\begin{estrofa}

\numero{17} \textparagraph\  Per dos estrems / Amor es mal mirent

\numero{18} per molt e poch / e lo mig se jaqueix

\numero{19} be's mostra pech / puix contra si falleix

\numero{20} car tot \c{c}o cau / que'strem es son tinent.

\numero{21} Bem marauell /si Amor no decau

\numero{22} puix que'n estrems /volt que'stiga son pes

\numero{23} lo meu voler / es mes que tota res

\numero{24} yl vostre's menys / q\textit{ue}l ter\c{c} d'vn pu\textit{n}t de dau.

\end{estrofa}



\begin{estrofa}

\numero{25} [12v] \textparagraph\  Durar no pot / sino mes fet gran frau

\numero{26} trencant Amor / de natura'ls costums

\numero{27} poch menys contrast /q\textit{ue} tenebres e llums

\numero{28} en mon voler /y el de ma dona jau.

\numero{29} O Deu perque /amor es des igual

\numero{30} que no consent / que vostre voler cresca?

\numero{31} perque lo meu / per null temps no peresca

\numero{32} si be nom'sent / quant me ve\textit{n}dr'aquest mal?

\end{estrofa}



\begin{estrofa}

\numero{33} Deu per bondat /vol ser tan cominal

\numero{34} que no consent /vn cor dur e saluatje

\numero{35} e ser amat /agran desauantatje

\numero{36} d'aquel qui es /en Amor son cabal.

\numero{37} Per \c{c}o nom pens / que Amor en mi dur

\numero{38} car en amar /vos e pres tot estrem

\numero{39} el vostre cor / es de amor axi sem

\numero{40} que'n mi pensar / no crech james atur.

\end{estrofa}


\begin{estrofaExtra}%Extra




\begin{tornada}

TORNADA.

\end{tornada}


\end{estrofaExtra}


\begin{estrofa}

\numero{41} Plena de seny / per lamor que vs port jur

\numero{42} que sim ve tart / la vostra ben volen\c{c}a

\numero{43} present de tots /fare de mi senten\c{c}a

\numero{44} que sonara / mentre'l mon dels vius dur.

\end{estrofa}



\begin{estrofaExtra}%Extra

\begin{final}

FIN DEL CANTO SEXTO.

\end{final}

\end{estrofaExtra}



\end{document}